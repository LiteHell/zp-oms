\documentclass{beamer}
\usepackage{kotex}
\usepackage{graphicx}
\usepackage{listings}
\usepackage{textcomp}

\lstset{
    language=Go,
    frame=single,
    breaklines=true,
    tabsize=2,
    basicstyle=\small
}

\title{Golang 써본 후기}
\author{신연진}
\date{2023년 11월 1일}
\logo{\includegraphics[height=0.5cm, keepaspectratio]{../zeropage_logo.png}}

\begin{document}
 \begin{frame}
    \maketitle
 \end{frame}

 \begin{frame}
  \tableofcontents
 \end{frame}

 \begin{frame}{들어가는 말}
    \section{들어가는 말}
    \begin{itemize}
     \item 기존에 운영되던 다음 사이트들은 Node.js로 개발되어 있었다.

     \begin{itemize}
      \item https://calendar.puang.network
      \item https://pre-timetable.puang.network
      \item https://rss.puang.network
     \end{itemize}

     \item Go언어가 괜찮다던데? 위 사이트들을 Go언어로 재작성해볼까?

    \begin{itemize}
      \item 그리하여 Go언어를 써보게 됐다.
     \end{itemize}
    \end{itemize}
 \end{frame}

 \begin{frame}{Go 언어?}
  \section{Go 언어}
  \subsection{소개}

    \begin{figure}
        \includegraphics[scale=5]{gopher.jpg}
        \caption{Go gopher}
    \end{figure}

  \begin{itemize}
   \item Google이 2009년에 개발한 언어
   \item 뇌비우고 코딩하기 좋은 언어
   \item 쓰다보면 묘하게 C언어 느낌이 좀 든다...
  \end{itemize}
 \end{frame}

 \begin{frame}{Go 언어의 장점}
 \subsection{장점}
  \begin{itemize}
   \item 추상화의 한계 -- 클래스 상속같은 개념이 없기 때문에 지나친 추상화로 인해 고통받을 일이 없다.
   \begin{itemize}
    \item C언어에서 struct와 함수로만 코딩하는 느낌. 다만 struct에 함수를 묶을 수는 있다.
   \end{itemize}

   \item 간단한 채널과 비동기 호출
   \begin{itemize}
    \item 함수 앞에 \textit{go} 키워드 하나만 비동기로 실행된다.
    \item 채널을 이용하면 비동기로 호출된 Task 간의 데이터 전달이 쉽다.
   \end{itemize}

   \item 컴파일 언어
   \begin{itemize}
    \item 당연하지만 인터프리터 언어보다 빠르다.
   \end{itemize}

   \item exception handling이 없다. 오류를 return value로 던진다(!)
  \end{itemize}

 \end{frame}

 \begin{frame}{명령어}
  \section{명령어}
  \begin{itemize}
   \item \textit{go mod init zeropage.org/example} -- 프로젝트 초기화
   \item \textit{go run} -- 실행
   \item \textit{go build} -- 빌드
   \item \textit{go test} -- 테스트
   \item \textit{go get example.com/example\_pkg} -- 패키지 설치
  \end{itemize}

 \end{frame}

 \begin{frame}[allowframebreaks,fragile]{Go언어 문법}
 \section{문법}
  \begin{itemize}
  \item 변수 선언
\begin{lstlisting}
 var test int // Define
 test = 1 // Assign

 test2 := 2 // Define and assign

 test3 = 3 // Error! test3 is not defined
\end{lstlisting}

    \item 형변환

    \begin{lstlisting}
var i int = 42
f := float64(i)
    \end{lstlisting}

    \framebreak

    \item 반복문
\begin{lstlisting}
 for i := 0; i < 5; i++ {
    fmt.Printf("%d\n", i)
 }
 j := 0
 for j < 100 {
    j += 1
    fmt.Printf("%d\n", j)
 }
 for {
    fmt.Printf("Infinitttttty")
 }
\end{lstlisting}

\framebreak

   \item 함수 선언
\begin{lstlisting}
 func example(a int, b int) int {
    return a + b
 }
\end{lstlisting}

    \item 조건문과 복수 개의 반환값
\begin{lstlisting}
 fnuc addPositives(a int, b int) (int, error) {
    if a <= 0 || b <= 0 {
        return nil, fmt.Errorf("They're not positives!'")
    }

    return a + b, nil
 }
\end{lstlisting}

    \framebreak

    \item 기명의 반환값

\begin{lstlisting}
 func plusTwoAndFour(a int, b int) (c, d int) {
    c = a + 2
    b = d + 2
    return
 }
\end{lstlisting}

    \item 패키지 선언 및 패키지 import
\begin{lstlisting}
 package main

 import (
    "fmt"
    "example.com/example_pkg"
 )
\end{lstlisting}

    \framebreak

    \item defer
\begin{lstlisting}
 func example() {
    defer fmt.Println(", world!")
    fmt.Print("Hello")
 }
\end{lstlisting}

    \item switch
    \begin{itemize}
     \item 참고: 조건 없으면 switch true랑 동일함
    \end{itemize}

\begin{lstlisting}
 switch a {
    switch 1:
    // Do something
    switch 2:
    // Do something
    default:
    //Do something
 }
\end{lstlisting}

  \end{itemize}
 \end{frame}

 \begin{frame}[fragile,allowframebreaks]{Go의 비동기 문법}
 \section{비동기 관련 문법}
  \begin{itemize}
   \item 비동기 호출
\begin{lstlisting}
func worker(a int) {
    for {
        // Do something
    }
}

func main () {
    for i := 0; i < 0; i++ {
        go worker(i)
    }
}
\end{lstlisting}
    \framebreak

    \item 채널
    \begin{lstlisting}[basicstyle=\tiny]
package main

import "fmt"

func sum(s []int, c chan int) {
	sum := 0
	for _, v := range s {
		sum += v
	}
	c <- sum // send sum to c
}

func main() {
	s := []int{7, 2, 8, -9, 4, 0}

	c := make(chan int)
	go sum(s[:len(s)/2], c)
	go sum(s[len(s)/2:], c)
	x, y := <-c, <-c // receive from c

	fmt.Println(x, y, x+y)
}
    \end{lstlisting}

    \framebreak

    \item range
    \begin{lstlisting}
func printer(c <-chan int) {
    for i := range c {
        fmt.Printf("%d", c)
    }
}

func main() {
    ch := make(chan int, 10)
    go printer(ch)
    go printer(ch)
    go printer(ch)
    for i := 0; i < 100; i++ {
        ch <- i
    }
}
    \end{lstlisting}


  \end{itemize}

 \end{frame}


 \begin{frame}{그래서 Go를 왜 쓰는가?}
 \section{그래서 Go를 왜 쓰는가?}
  \begin{itemize}
   \item 단순한 문법
   \begin{itemize}
    \item 단순하다 \textrightarrow 무식하게 코딩하기 딱 좋다
   \end{itemize}

   \item 테스트 프레임워크 기본 내장
   \begin{itemize}
    \item 테스트 코드 짜기 편하다
   \end{itemize}

   \item 동시성
   \begin{itemize}
    \item 함수 호출 앞에 \textit{go}만 붙이면 비동기 호출이 된다고?
   \end{itemize}

   \item 안 쓰는 변수가 있을 시 오류 발생
   \begin{itemize}
    \item 호불호가 갈리는 부분이지만, 실수를 예방해준다는 측면으로 보면 좋은 점이다.
   \end{itemize}

  \end{itemize}
 \end{frame}

\end{document}
